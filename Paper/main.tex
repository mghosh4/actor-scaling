% TEMPLATE for Usenix papers, specifically to meet requirements of
%  USENIX '05
% originally a template for producing IEEE-format articles using LaTeX.
%   written by Matthew Ward, CS Department, Worcester Polytechnic Institute.
% adapted by David Beazley for his excellent SWIG paper in Proceedings,
%   Tcl 96
% turned into a smartass generic template by De Clarke, with thanks to
%   both the above pioneers
% use at your own risk.  Complaints to /dev/null.
% make it two column with no page numbering, default is 10 point

% Munged by Fred Douglis <douglis@research.att.com> 10/97 to separate
% the .sty file from the LaTeX source template, so that people can
% more easily include the .sty file into an existing document.  Also
% changed to more closely follow the style guidelines as represented
% by the Word sample file. 

% Note that since 2010, USENIX does not require endnotes. If you want
% foot of page notes, don't include the endnotes package in the 
% usepackage command, below.

% This version uses the latex2e styles, not the very ancient 2.09 stuff.
\documentclass[letterpaper,twocolumn,10pt]{article}

\usepackage{graphicx}
\usepackage{subcaption}
\usepackage[hyphens]{url}
\usepackage{amsthm, amsmath, amssymb}
\usepackage{color}
\usepackage{paralist}
\usepackage[boxed,noline]{algorithm2e}
\usepackage{enumitem}
\usepackage{flushend}

\usepackage{macros}
\usepackage{usenix,epsfig,endnotes}

\begin{document}

%don't want date printed
\date{}

%make title bold and 14 pt font (Latex default is non-bold, 16 pt)
\title{\Large \bf Efficient Scaling of Stateful Actors}

%for single author (just remove % characters)
\author{
%{\rm
% copy the following lines to add more authors
% \and
% {\rm Name}\\
%Name Institution
} % end author

\maketitle

% Use the following at camera-ready time to suppress page numbers.
% Comment it out when you first submit the paper for review.
\thispagestyle{empty}

%\input{abstract.tex}

\input{texfiles/;intro.tex}

%\input{problem.tex}

%\input{simulation.tex}

%\input{related-work.tex}

%\input{ack.tex}

%\input{summary.tex}

{\footnotesize \bibliographystyle{acm}
\bibliography{bibliography}}

%\appendix

%\input{appendix_optimality.tex}

\end{document}