% TEMPLATE for Usenix papers, specifically to meet requirements of
%  USENIX '05
% originally a template for producing IEEE-format articles using LaTeX.
%   written by Matthew Ward, CS Department, Worcester Polytechnic Institute.
% adapted by David Beazley for his excellent SWIG paper in Proceedings,
%   Tcl 96
% turned into a smartass generic template by De Clarke, with thanks to
%   both the above pioneers
% use at your own risk.  Complaints to /dev/null.
% make it two column with no page numbering, default is 10 point

% Munged by Fred Douglis <douglis@research.att.com> 10/97 to separate
% the .sty file from the LaTeX source template, so that people can
% more easily include the .sty file into an existing document.  Also
% changed to more closely follow the style guidelines as represented
% by the Word sample file. 

% Note that since 2010, USENIX does not require endnotes. If you want
% foot of page notes, don't include the endnotes package in the 
% usepackage command, below.

% This version uses the latex2e styles, not the very ancient 2.09 stuff.
\documentclass[letterpaper,twocolumn,10pt]{article}

\usepackage{graphicx}
\usepackage{subcaption}
\usepackage[hyphens]{url}
\usepackage{amsthm, amsmath, amssymb}
\usepackage{color}
\usepackage{paralist}
\usepackage[boxed,noline]{algorithm2e}
\usepackage{enumitem}
\usepackage{flushend}

\usepackage{macros}
\usepackage{usenix,epsfig,endnotes}

\begin{document}

%don't want date printed
\date{}

%make title bold and 14 pt font (Latex default is non-bold, 16 pt)
\title{\Large \bf Efficient Scaling of Stateful Actors}

%for single author (just remove % characters)
\author{
%{\rm
% copy the following lines to add more authors
% \and
% {\rm Name}\\
%Name Institution
} % end author

\maketitle

% Use the following at camera-ready time to suppress page numbers.
% Comment it out when you first submit the paper for review.
\thispagestyle{empty}

%\input{abstract.tex}

\section{Thoughts}
\label{sec:intro}

\subsection{Problem Statement}

Actor Frameworks: Actors provide a shared nothing abstraction that allows horizontal scaling. Scaling stateful actors is a challenge because of state reconciliation overheads. The problem has similarities with storage systems where replication is necessary to scale read/write performance. Some key differences are:
\todo{Let's define key terms before using to make things clear: actor &  stateful(and stateless) -- talk about mixing computation and
state}

\todo{frame actors in orleans: virtual actors, silos, etc}

1) In a LAN setting, the number of instantiated actors can be much larger than the total number of read/write replicas -- Can we show some plots based on data?
\todo{How would we do this? Add replicas in a k-v or DB and show they don't scale as well as an actor framework?}
2) \todo{Execution / threading model is different. A DB will typically have many threads/workers reading and writing on the same
item. For main memory systems this is often bound by the number of threads/cores on the physical server.  For Orleans,
the execution of an Actor is single threaded (even thought it is not pinned to a core). }


Given a latency SLA and read write requirement, we need to decide the following things: 1) Whether to instantiate a new actor, and 2) Consistency level of the new actor
\todo{expand on our SLA and r/w requirements.}

We can probably break down the problem. In the first instance the user can provide client SLAs with both consistency and latency similar to Pileus and we need to propose an algorithm which decides whether to instantiate an actor or not. As a next step, we can remove consistency from the SLA and see if we can decide the consistency from the latency SLA.

This work is different from Pileus because it does not handle replica creation, leader election which our work needs to consider in an online manner

In ``Take me to your leader'' work, given a set of possible sites, it tries to figure out replica locations, roles and leader. This calculation happens periodically and the replica roles are changed. They do not consider SLAs though.
\todo{Maybe for now we can make a table like the google drive doc and differentiate against each paper in terms of setting, constraints
target objective, and simple summary of approach}

\subsection{Paper Format}


%\input{problem.tex}

%\input{simulation.tex}

%\input{related-work.tex}

%\input{ack.tex}

%\input{summary.tex}

{\footnotesize \bibliographystyle{acm}
\bibliography{bibliography}}

%\appendix

%\input{appendix_optimality.tex}

\end{document}